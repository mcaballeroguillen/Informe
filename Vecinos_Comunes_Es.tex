Este método fue utilizado  \cite{newman2001clustering}  en el estudio de redes de colaboración, mostrando una correlación positiva entre el número de vecinos comunes y la probabilidad de que dos científicos colaboren en el futuro; concluyendo que la probabilidad de colaboración entre científicos  en función de su número de conocidos mutuos en la red está fuertemente correlacionada positivamente.